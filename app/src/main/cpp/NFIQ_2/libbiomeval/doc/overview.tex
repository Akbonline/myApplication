\chapter{Overview}

The \lname (\sname)
is a set of C++\cite{cpp:plguide} classes, error codes, and design
patterns used to create a common environment to provide logging, data
management, error handling, and other functionality that is needed for many
applications used in the testing of biometric software. The goals of the
framework include:
\begin{itemize}
\item Reduce the amount of I/O error handling implemented by applications.
\item Provide standard interfaces for data management and logging;
\item Remove the need for applications to handle low-level events from the
operating system (signals, etc.);
\item Provide services for timing the execution of code blocks;
\item Allow applications to constrain the amount of processing time used
by a block of code;
\item Reduce memory allocation errors;
\item Simplify the use of parallel processing.
\end{itemize}

The experience of the \nistig when running many software evaluations has led
to the need of a common code for dealing with recurring software issues. One
issue is the large amounts of data consumed, and created, by the software
under test. Input data sets are typically biometric images, while output sets
contain derived information. Both sets of data often contain millions of
items, and storing each item as a file creates a tremendous burden on the file
system. The \namespace{IO} package provides a solution to
managing large amounts of records in a portable, efficient manner, as well as 
facilities for logging and maintaining runtime settings.

\sname is divided into several packages, each providing a set of
related functionality, such as error handling and timing operations. The
packages are an informal concept, mapped to formal C++ name spaces, e.g.
\namespace{IO} and \namespace{Time}. A namespace contains classes, constants,
and non-class functions that relate to concepts grouped in the namespace.
All classes within \sname belong to the top-level \namespace{BiometricEvaluation}
namespace.

Biometric image data is often supplied in a compressed format (e.g. WSQ, JPEG)
and must be converted to a ``raw'' format. The \namespace{Image} package contains
classes to represent compressed image data as an object, storing the image
size and other attributes, in addition to the raw image.

Memory management issues are addressed by the \namespace{Memory} package. The
use of classes and templates in this package can relieve applications of the
need to directly manage memory for dynamically sized arrays, or call functions
that are already provided to allocate and free C library objects.

While a program is running, it is often necessary to record certain statistics
about the process, such as memory and processor usage. The \namespace{Process}
package provides methods to obtain this information, as well as the capability
to log to a file periodically, in an asynchronous manner.

In addition to its own statistics, a program may need to query some
information about the environment under which it is running.
The \namespace{System}
package provides a count of CPUs, memory size, other system characteristics
that an application can use to tailor its behavior.

Many aspects of software performance evaluation involve the use of timers. The
\namespace{Time} package provides for the calculation of a time interval in a manner
that is consistent across platforms, abstracting the underlying operating
system's timing facility. Also, included is a ``watchdog'' facility, providing
a solution to the problem of non-returning function calls. By using a watchdog
timer, an application can abort a call to a function that doesn't return in
the required interval.

The \namespace{Text} package provides a set of utility functions for operating on
strings. The \code{digest} functions are of interest to
those applications that must mask any information contained in a string before
passing that information to another function. For example, often the biometric
image file (or record) names contain information about the image, such as the
finger position.

Error propagation and handling are addressed by the \namespace{Error} package. A set
of exception objects are defined within this package, allowing for communication
of error conditions out of the framework to the application, along with an
explanatory string. Signal handling is related to error propagation in that
when a process receives a signal, often it is due to software bug. Divide by
zero, for example. The \namespace{Error} package provides for simple handling
of the signal by the process.

Many packages in \sname deal with biometric data record formats, including
ANSI/NIST~\cite{std:an2k} records. In order to provide a general interface
to several formats, \sname represents the biometric data as derived from
a source. For example, the \namespace{Finger} package contains classes that represent
all information about a finger, including the source image and derived
minutiae points. The \namespace{View} package combines the notions of a source
image and derived information together into a single abstraction.

Applications can use the \namespace{Messaging} package to communicate between
threads and processes, or to a terminal. Messages in this context are simply
an array of bytes. One such use could be providing a command line interface
to an long-running process.

The \namespace{MPI} package provides wrappers around the Message Passing
Interface~(MPI)~\cite{mpi} libraries, handling all MPI communcation and
error events. Many parallel applications can be greatly simplified, only
implementing a few methods to process data.

\sname is designed to be used in a modular fashion, and it is possible to
compile many packages independently. However, several packages do make use
of other packages in the framework, and therefore, are less flexible in their
reuse. However, \sname is designed to reduce the intra-framework
dependencies.

A set of test programs is included with the framework. These programs not only
exercise the functions provided by the packages, but also can be used as
example programs on how to use framework.

The chapters that follow this overview describe each package in detail,
along with some code examples.
The final set of chapters of this document contain the application programming
interfaces for the types, methods, and classes that make up \sname. However,
the framework is under development, and other packages, classes, etc. will be
added over time to address the needs of the \nistig.
