%
% Data Interchange API
%
\chapter{Data Interchange}
\label{chp-datainterchange}
The \namespace{DataInterchange} package consists of classes and other elements
used to
process an entire biometric data record, or set of records. For example,
a single ANSI/NIST 
record, consisting of many smaller records (fingerprint images, latent data,
etc.) can be accessed by instantiating a single object. Classes in this
package typically use has-a relationships to classes in the \namespace{Finger}
and other packages that process individual biometric samples.

The design of classes in the \namespace{DataInterchange} package allows
applications to
create a single object from a biometric record, such as an ANSI/NIST file.
After creating this object, the application can retrieve the needed information
(such as finger views~\chpref{chp-finger}) from this object.
A typical example would be to retrieve all
images from the record and pass them into a function that extracts a biometric
template or some other image processing.

\section{ANSI/NIST Data Records}
\label{sec-ansinistdatarecords}

The ANSI/NIST Data Interchange package contains the classes used to represent
ANSI/NIST~\cite{std:an2k} records. One class, \class{AN2KRecord},
is used to represent the entire ANSI/NIST record. An object of this class
will contain objects of the \class{Finger} classes, as well as other packages.
By instantiating the \class{AN2KRecord} object, the application can retrieve all
the information and images contained in the ANSI/NIST record.

The \class{AN2KMinutiaeDataRecord} class represents an entire Type-9 record
from an ANSI/NIST file. However, some components of this class are represented 
by classes in other packages. For example, the \class{AN2K7Minutiae} class in
the \namespace{Feature}
package represents the ``standard'' format minutiae in the Type-9 record.

\lstref{lst:an2kdiuse} shows how an application can retrieve all finger
latents (Type-13) and captures (Type-14) from an ANSI/NIST record. Also shown
is the general record information such as the capture date, etc.
Once the views are
retrieved, the application obtains the set of minutiae records associated with
that view. In addition, the example shows how the entire set of minutiae
records can be read independent of a view.

\lstref{lst:an2kefsuse} shows how to retrieve the extended feature set data by
constructing a data interchange object.

\begin{lstlisting}[caption={ANSI/NIST Data Interchange}, label=lst:an2kdiuse]
#include <iostream>
#include <be_data_interchange_an2k.h>

/*
 * This test program exercises the Evaluation framework to process an AN2K
 * records stored in a file. The intent is to model what a real program
 * would do by retrieving AN2K records, doing some processing on the image,
 * and displaying the results.
 */
using namespace std;
using namespace BiometricEvaluation;
using namespace BiometricEvaluation::Framework::Enumeration;

static void
printRecordInfo(const DataInterchange::AN2KRecord &an2k)
{
	cout << "\tVersion: " << an2k.getVersionNumber() << endl;
	cout << "\tDate: " << an2k.getDate() << endl;
	cout << "\tDestination Agency: " <<
	    an2k.getDestinationAgency() << endl;
	cout << "\tOriginating Agency: " <<
	    an2k.getOriginatingAgency() << endl;
	cout << "\tTransaction Control Number: " <<
	    an2k.getTransactionControlNumber() << endl;
	cout << "\tNative Scanning Resolution: " <<
	    an2k.getNativeScanningResolution() << endl;
	cout << "\tNominal Transmitting Resolution: " <<
	    an2k.getNominalTransmittingResolution() << endl;
	cout << "\tCapture Count: " << an2k.getFingerCaptureCount() << endl;
	cout << "\tLatent Count: " << an2k.getFingerLatentCount() << endl;
}

static void
printViewInfo(const View::AN2KViewVariableResolution &an2kv)
{
	cout << "\tRecord Type: " <<
	    static_cast<std::underlying_type<
	    View::AN2KView::RecordType>::type>(an2kv.getRecordType()) << endl;
	cout << "\tImage resolution: " << an2kv.getImageResolution() << endl;
	cout << "\tImage size: " << an2kv.getImageSize() << endl;
	cout << "\tImage color depth: " << an2kv.getImageColorDepth() << endl;
	cout << "\tCompression: " <<
	    to_string(an2kv.getCompressionAlgorithm()) << endl;
	cout << "\tScan resolution: " << an2kv.getScanResolution() << endl;
	cout << "\tImpression Type: " << to_string(an2kv.getImpressionType()) <<
	    endl;
	cout << "\tSource Agency: " << an2kv.getSourceAgency() << endl;
	cout << "\tCapture Date: " << an2kv.getCaptureDate() << endl;
	cout << "\tComment: [" << an2kv.getComment() << "]" << endl;

	/*
	 * Get the image data.
	 */
	auto img = an2kv.getImage();
	if (img != nullptr) {
		/* Do something with the image info and data */
		;
	} else {
		cout << "No Image available.\n";
	}

	/*
	 * Print info for the minutiae associated with this view.
	 */
	auto minutiae = an2kv.getMinutiaeDataRecordSet();
	cout << "\tThere are " << minutiae.size() << 
	    " minutiae data records.\n";
}

int
main(int argc, char* argv[]) {
	try {
		DataInterchange::AN2KRecord an2k("test_data/a002.an2");
		printRecordInfo(an2k);
		/*
		 * Obtain the finger capture and latent views from the
		 * AN2k file.
		 */
		int i = 0;
		for (auto c: an2k.getFingerCaptures()) {
			cout << "[Capture View " << i++ <<"]\n";
			printViewInfo(c);
			cout << "\tPosition: " << c.getPosition()
			    << endl;
			cout << "[End of Capture View]\n";
		}
		i = 0;
		for (auto l: an2k.getFingerLatents()) {
			cout << "[Latent View " << i++ <<"]\n";
			printViewInfo(l);
			cout << "\tPositions: ";
			for (auto p: l.getPositions()) {
				cout << p << " ";
			}
			cout << endl << "[End of Latent View]\n";
		}
		/*
		 * Obtain the entire set of minutiae records from the
		 * AN2k file, independently of the view.
		 */
		auto minutiae = an2k.getMinutiaeDataRecordSet();
		cout << "There is a total of " << minutiae.size()
		    << " minutiae data records in the AN2K file.\n";
		cout << ">>>>>>>>>>>>>>>>>>>>>>>>>>>>>>>>>>>>>>>>\n";
	} catch (Error::Exception &e) {
		cout << "Failed sequence: " << e.what() << endl;
		return (EXIT_FAILURE);
	}
}
\end{lstlisting}

\section{INCITS Data Records}
\label{sec-incitsdatarecords}

The INCITS class of data records covers all those record formats that are
derived from the standards defined by the InterNational Committee for
Information Technology Standards~\cite{org:incits}. These formats include the
ANSI-2004 Finger Minutiae Record Format~\cite{std:ansi378-2004}, the ISO
equivalent~\cite{std:iso19794-2}, and other data formats, including finger
images.

The \class{DataInterchange::ANSI2004Record} represents all the finger views
contained in a pair of ANSI 2004 fingerprint(\cite{std:ansi378-2004}) and
finger image (\cite{std:ansi381-2004}) records. This class supports the
insert/update/remove of finger views from the data interchange record,
enabling the runtime updating of the object. In addition, the encoded format
of the minutia record can be obtained, enabling the read/modification/write
of the record.

\warn{Reading data from finger image records is not currently supported}

\begin{lstlisting}[caption={ANSI 2004 Data Interchange}, label=lst:ansi2004diuse]
#include <iostream>
#include <be_data_interchange_ansi2004.h>

using namespace std;
using namespace BiometricEvaluation;
using namespace BiometricEvaluation::Framework::Enumeration;

void
printViewInfo(Finger::INCITSView &fngv)
{
	cout << "Begin ------------------------------------------" << endl;
	cout << "Image resolution is " << fngv.getImageResolution() << endl;
	cout << "Image size is " << fngv.getImageSize() << endl;
	cout << "Image depth is " << fngv.getImageColorDepth() << endl;
	cout << "Compression is " << fngv.getCompressionAlgorithm() << endl;
	cout << "Scan resolution is " << fngv.getScanResolution() << endl;

	cout << "Finger position is " << fngv.getPosition() << endl;
	cout << "Impression type is " << fngv.getImpressionType() << endl;
	cout << "Quality is " << fngv.getQuality() << endl;
	cout << "Eqpt ID is " << hex << showbase << fngv.getCaptureEquipmentID() << endl;
	cout << dec;

	Feature::INCITSMinutiae fmd = fngv.getMinutiaeData();
	cout << "Minutiae format is " << to_string(fmd.getFormat()) << endl;
	cout << "There are " << fmd.getMinutiaPoints().size()
	     << " minutiae points." << endl;
	cout << "End ------------------------------------------" << endl;
}

bool
showAllViews(const DataInterchange::ANSI2004Record &record)
{
	if (record.getNumFingerViews() == 0) {
		cout << "No finger views present.\n";
		return (true);
	}
	for (int i = 1; i <= record.getNumFingerViews(); i++) {
		cout << "++++++++++++++++++++++++++++++\n";
		cout << "View number " << i << ":\n";
		auto fngv = record.getView(i);
		printViewInfo(fngv);
		cout << "Test getMinutia(): View " << i << " has "
		    << record.getMinutia(i).getMinutiaPoints().size()
		    << " minutiae points.\n";
	}
	return (true);
}

int
main(int argc, char* argv[])
{
	std::unique_ptr<DataInterchange::ANSI2004Record> record;

	/* Construct with a file, minutia record only. */
	try {
		record.reset(new DataInterchange::ANSI2004Record(
		    "test_data/fmr.ansi2004", ""));
	} catch (Error::Exception& e) {
		cout << "A file error occurred: " << e.what() << endl;
		return (EXIT_FAILURE);
	}

	/* Remove all views but the first */
	record->isolateView(1);
	showAllViews(*record);

	/* Modify the minutia in a finger view */
	auto minutiaRecord = record->getMinutia(1);
	auto minutiaPoints = minutiaRecord.getMinutiaPoints();
	for (auto& fm: minutiaPoints) {
		fm.coordinate.x += 10;
		fm.coordinate.y += 10;
	}
	/* Replace minutiae in the remaining view */
	minutiaRecord.setMinutiaPoints(minutiaPoints);
	record->setMinutia(1, minutiaRecord);
	showAllViews(*record);

	/* Obtain the ANSI-378 record and instantiate an object from it */
	auto fmr = record->getFMR();
	BE::Finger::ANSI2004View fmrView(fmr, Memory::uint8Array{}, 1);
	/* The fmr object can also be written to a file */
	
	return (EXIT_SUCCESS);
}
\end{lstlisting}
