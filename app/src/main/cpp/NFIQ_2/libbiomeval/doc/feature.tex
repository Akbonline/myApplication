%
% Feature API
%
\chapter{Feature}
\label{chp-feature}
The \namespace{Feature} package contains those items that relate to the representation of
biometric features, such as fingerprint minutiae, facial features (eyes, etc.),
and related information. Objects of these class types are typically associated
with \namespace{View}~(\chpref{chp-view}) or
\namespace{DataInterchange}~(\chpref{chp-datainterchange})
objects. For example, a minutiae object is usually obtained from a finger view,
which may have been obtained from a data interchange object representing an
entire biometric record for an individual.

The data contained within a \class{Feature} object is represented as the
``native'' format as it was extracted from the underlying data record. There
is no translation to a common format and it is the application's responsibility 
to interpret or translate the data as necessary.

Currently, fingerprint and palm print minutiae are the features supported within
the \sname. As development continues, additional features contained within
biometric data records will be supported.

\section{ANSI/NIST Features}
\label{sec-ansinistfeatures}
The ANSI/NIST~\cite{std:an2k} standard defines several features represented
as data elements within a record. Fingerprint and palm minutiae is contained
within Type-9 record. The \class{AN2K7Minutiae} class, contained in the Feature
package, represents a single Type-9 record. An object of this class can be
constructed directly from a complete ANSI/NIST record. However, it is more
common for an application to retrieve these objects from the \class{AN2KView}
object defined in the \namespace{Finger} package~(\chpref{chp-finger}).

See \lstref{lst:an2kfingerviewuse} for a complete example of how to obtain
the fingerprint minutiae data from an ANSI/NIST record. If only extended
feature set data is required from the file, a
\class{Feature::AN2K11EFS::ExtendedFeatureSet} object can be created directly
from the file or memory buffer.

\subsection{ANSI/NIST 2011 Extended Feature Sets}
The 2011 edition of the ANSI/NIST standard~\cite{std:an2k11} adds a new form
of feature data representation to the Type-9 fingerprint minutiae record. The
extended feature set information is represented by an object that can be
retrieved from the \class{AN2KMinutiaeDataRecord} object created from the data
file.

\lstref{lst:an2kefsuse} shows how to read the extended feature set
data from an ANSI/NIST file, both as a data interchange object
(see \secref{chp-datainterchange}) or an extended feature set object constructed
directly from a file.

\begin{lstlisting}[caption={Using AN2K Extended Feature Sets}, label=lst:an2kefsuse]
#include <iostream>
#include <be_data_interchange_an2k.h>
#include <be_feature_an2k11efs.h>

/*
 * This test program exercises the Evaluation framework to process AN2K
 * records stored in a RecordStore. The intent is to model what a real
 * program would do by retrieving AN2K records, doing some processing
 * on the image, and displaying the results.
 */
using namespace BiometricEvaluation;

static void
printAN2K11EFS(Feature::AN2K11EFS::ExtendedFeatureSet &efs)
{
	Image::ROI roi = efs.getImageInfo().roi;
	std::cout << "ROI:\n"
	    << "\tSize: ("
	    << roi.size.xSize << "," << roi.size.ySize << ")\n"
	    << "\tOffset: ("
	    << roi.horzOffset << "," << roi.vertOffset << ")\n"
	    << "\tPath: ";
	for (auto const& point: roi.path) {
		std::cout << point << " ";
	}
	std::cout << "\n";

	std::cout << "Image Info:\n" << efs.getImageInfo() << "\n\n";

	Feature::AN2K11EFS::CorePointSet cps = efs.getCPS();
	std::cout << "CPS: Have " << cps.size() << " EFS core point(s):\n";
	for (auto const& cp: cps) {
		std::cout << "\t" << cp << "\n";
	}

	Feature::AN2K11EFS::DeltaPointSet dps = efs.getDPS();
	std::cout << "DPS: Have " << dps.size() << " EFS delta point(s):\n";
	for (auto const& dp: dps) {
		std::cout << "\t" << dp << "\n";
	}

	Feature::AN2K11EFS::MinutiaPointSet mps = efs.getMPS();
	std::cout << "MPS: Have " << mps.size() << " EFS minutia point(s):\n";
	for (auto const& mp: mps) {
		std::cout << mp << "\n";
	}

	std::cout << "No Features Present:\n";
	std::cout << efs.getNFP();

	std::cout << "\nMinutiae Ridge Count Information:\n";
	auto mrci = efs.getMRCI(); 
	std::cout << mrci << "\n";
}

int
main(int argc, char* argv[]) {

	std::string fname = "test_data/type9-efs.an2k";
	/*
	 * Read the EFS data from the DataInterchange::AN2KRecord object
	 */
	std::cout << "Extended Feature Set data in " << fname << ": ";
	try {
		DataInterchange::AN2KRecord an2k(fname);
		std::vector<Finger::AN2KMinutiaeDataRecord> minutiae =
		    an2k.getMinutiaeDataRecordSet();
		printAN2K11EFS(*minutiae[0].getAN2K11EFS());
	} catch (Error::Exception &e) {
		std::cout << "Failed; caught " << e.whatString() << "\n";
	}

	/*
	 * Read the EFS data by constructing directly from the filename
	 */
	try {
		Feature::AN2K11EFS::ExtendedFeatureSet efs(fname, 1);
		printAN2K11EFS(efs);
	} catch (Error::Exception &e) {
		std::cout << "Failed; caught " << e.whatString() << "\n";
	}
}
\end{lstlisting}

\section{ISO/INCITS Features}

The ISO~\cite{std:iso19794-2} and INCITS~\cite{std:ansi378-2004} fingerprint
minutiae standards are represented within \sname with the same class,
\class{INCITSMinutiae}, as the minutiae format is identical in both standards.

\lstref{lst:incitsfingerviewuse} shows how to create a view object for the
fingerprint minutiae record contained in a file.
